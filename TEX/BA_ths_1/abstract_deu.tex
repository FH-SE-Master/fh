\chapter{Kurzfassung}
Die vorliegende Arbeit setzt sich mit den heuristischen Algorithmen der Klassifikation und  des Clustering auseinander. \\ Das Clustering, welches auch als unüberwachte Klassifikation bekannt ist, wird als eine Methode zur Einteilung von großen Datensätzen angewandt. Hier wird auf den unterschiedlichen Bezug der Daten  auf das Clustering und dessen Algorithmen eingegangen. Verschiedene Algorithmen geben einen guten Einblick in die komplexe Welt des Clustering. Grundsätzlich gilt, dass jede Methode ihr eigenes Einsatzgebiet innehat und an die vorliegenden Daten anzupassen ist.\\ 
In weiterer Folge wird mit der Klassifikation eine weitere Methode beschrieben Daten einzuteilen. Im Gegensatz zum Clustering werden bei Klassifikation Daten anhand vorgegebener Klassen zugeordnet. \\
Auch hier wurden die verschieden Algorithmen unter der Berücksichtigung der Klassifikationsgüte beschrieben und verglichen. Alle Algorithmen verwenden bereits vordefinierte Daten und Methoden um die Berechnung zu beschleunigen. Daher ist die Klassifikation weiterverbreiteter als das Clustering. \\ Es wurden die Algorithmen und Methoden anhand von  Beispielen  entsprechend dargestellt. Dabei kamen das Heuristiclabtool sowie Daten aus UCI Repository zur Anwendung. Diese Beispiele zeigen ziemlich gut welche Algorithmen für welche Datensätze geeignet sind und beste Ergebnisse liefern.
