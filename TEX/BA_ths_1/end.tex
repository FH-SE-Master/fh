
\chapter{Schluss}
\section{Zusammenfassung}
In dieser Arbeit wurden die Themen Clustering und Klassifikation behandelt. Ziel war es die verschieden Algorithmen aufzuzeigen und zu vergleichen. Grundlegend wird von Klassifikation ausgegangen, die sich in eine überwachte und unüberwachte Klassifikation einteilen lässt. Bei der unüberwachten Klassifikation kann auch von Clustering gesprochen werden, da es hier keine eindeutige Klasseneinteilung gibt. Im Gegensatz dazu ist bei der klassischen Klassifikation die Einteilung bzw. die Zuordnung bekannt. Doch arbeiten beide Verfahren nach demselben Prinzip. Umfangreiche Daten werden in verwendbare Teile eingeteilt.\\
Die beiden Methoden spielen in der heutigen Welt eine große Rolle und werden häufig in der Datenanalyse sowie im Bereich des Data Mining verwendet. Daher sind unterschiedliche Algorithmen implementiert worden, um dieses Problem zu lösen. Dabei kommt es auf die verwendeten Daten an, welcher Algorithmus die besten Ergebnisse liefert.\\
Abschließend wird zusammengefasst, dass in dieser Arbeit nicht alle Algorithmen beschrieben worden sind. In dieser Arbeit wurden nur die bedeutsamsten und die relevanten Algorithmen aufgelistet. Die Algorithmen werden nicht nur für Clustering und Klassifikation verwendet, sondern auch in Bereichen wo Datenanalysen und Heuristiken eine wesentliche Rolle spielen.

