\chapter{Kurzfassung}
Diese Arbeit setzt sich mit der Implementierung eines Moduls für MRP mit der EWS API auseinander. Das MRP-System generiert mit Hilfe von Ressourcen und Regeln Termine, welche in das Microsoft Outlook übertragen werden. Ziel ist es die Verbindung von MRP und Microsoft Outlook herzustellen. Das Modul ist in JAVA programmiert. Mit Hilfe der EWS API wird die Verbindung zum Server geschaffen.\\ Bei der Implementierung wird geachtet, dass die Richtlinien von der Zielsetzung eingehalten werden. Es wird ein Konzept zur Ansteuerung mittels MRP vorgestellt. In der Implementierung werden die Funktionalitäten sowie Methoden beschrieben, die verwendet worden sind. Weiters werden in der Implementierung die EWS Funktionen angewendet und an das MRP angepasst. Die einzelnen Funktionen werden mittels der Methode des benutzerinteraktiven Testens überprüft. Die verschiedenen Funktionalitäten werden getestet und die Ergebnisse diskutiert.
