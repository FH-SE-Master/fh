\chapter{Einleitung}
Diese Arbeit ist der praktische Teil der Bachelorarbeit.  Diese wurde im Rahmen des Semesterpraktikums vom 01.03 bis 30.06.2016 verfasst. Das Praktikum im Umfang von 16 Wochen wurde bei der Firma CGM Clinical Austria am Standort Linz absolviert. CompuGroup Medical (CGM) ist ein österreichischer Softwarehersteller, der umfassende IT-Lösungen zur Optimierung des Gesundheitswesens produziert. Software-Lösungen, welche die Prozesse von niedergelassenen Ärzten und deren Personal sowie von medizinischen, pflegerischen und administrativen Krankenhauspersonal unterstützen. Die CGM beschäftigt in Österreich ca. 250 Mitarbeiter (\url{https://www.cgm.com/at/index.de}).

\section{Zielsetzung}
Das Modul Multidimensionale Ressourcen Planning (MRP) ermittelt Termine für Operationen in Krankenhäusern unter der Berücksichtigung der beteiligten Ressourcen (Personen, Geräte, Material) und vor und nachgelagerten Untersuchungen inklusive Betten und Zimmern. Ziel dieses Projektes ist es für die ressourcenbezogenen Daten (z.B. Dienste von Ärzten , Operationstermine etc.) eine Möglichkeit zu schaffen, Termine in den persönlichen Kalendern der beteiligten Personen (Teams) anzuzeigen. Damit sollen die Anwender in ihrer gewohnten Umgebung die Ergebnisse des komplexen Planungsprozesses des MRP-Systems vermittelt bekommen. Es wird davon ausgegangen, dass Termine in Microsoft Outlook eingetragen werden. Dies soll automatisch nach der Generierung von vorgegebenen Terminvorschlägen geschehen. Diese Terminvorschläge werden als Teil des MRP-Systems generiert. Die Aufgabe war es mit Hilfe der Java Exchange Web Services  (EWS) Application Programming Interface (API) die Daten an das  Microsoft Outlook zu senden bzw. an den Server zu übermitteln. Die EWS API erleichtert das Arbeiten mit  Microsoft Outlook.\newpage Bei flexibler API, kann sowohl Java, C\# als auch Extensible Markup Language (XML) verwendet werden. Die Termine werden dabei in Microsoft Outlook-Kalendern dargestellt. Änderungen von Terminen im Microsoft Outlook haben keine Auswirkung auf das MRP System. Weiters können sogenannte "`Appointments"' mit Hilfe von vordefinierten Filtern gesucht und exportiert werden. Dies ermöglicht eine einfachere Handhabung von Abfragen. Die Appointments werden dann in Outlook-Termine umgewandelt und können so exportiert werden. 