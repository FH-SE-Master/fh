\chapter {Zusammenfassung}
In diesem Kapitel werden die Ergebnisse zusammengefasst und diskutiert und mögliche Verbesserungsvorschläge angezeigt.
\section{Resultate}
Bei dieser Arbeit wurde ein Programm implementiert , welches als Modul für das MRP-System (Version 3) der Firma CGM vorgesehen war. Dieses Modul beschäftigte sich mit der  Ansteuerung eines Outlookservers mit Hilfe der EWS-API. Die Implementierung wurde mittels JAVA durchgeführt und daher wurde Java EWS API verwendet. Die EWS-API dient zur Ansteuerung eines Outlookservers mithilfe von Programmmethoden. Das Modul verwendete diese API um bereitgestellte Termine, welche Ressourcendaten aus dem MRP-System beinhalten können an das Outlook exportieren zu können. Diese resourcenbezogenen Daten wurden von anderen Modulen des MRP-System generiert und bereitgestellt. Allerdings fehlte die Integration des Moduls in das gesamtheitliche System des MRP, da dieses System zum derzeitigen Zeitpunkt noch nicht vollständig fertig gestellt war. Dabei wurde keine vollständige Testumgebung verwendet, da das Testen nur teilweise möglich war, da das Modul nicht in dem Testumfang integriert war. Das Testen erfolgte durch benutzerorientiertes Testen. Dieses beruhte auf eine rein visuelle Anschauung des jeweiligen Programmierers. Alle diesbezüglichen Tests lieferten am Ende der Arbeit ein positives Ergebnis.
\section{Diskussion}
Das in dieser Arbeit erstellte Programmmodul ist universell einsetzbar, wurde aber geringfügig an das MRP-System angepasst. Die komplexen Datentypen die die EWS verwendet, wurden durch das vereinfachte DTO ersetzt, Die universelle Einsetzbarkeit wurde damit noch zusätzlich verbessert. Weiters wurde die XML-Konformität implementiert. Dies bedeutet, dass XML-Dokumente gelesen und geschrieben werden können. Die Implementierung der verwendeten Filter wurde generisch gehalten, um eine universelle Einsetzbarkeit zu ermöglichen.
\section{Ausblick}
Da das MRP-Modul noch nicht vollständig implementiert wurde , wird erwartet, dass dieses in den nächsten Jahren von der Firma CGM fertiggestellt wird und somit eine Integration des Outlook erfolgen kann. Das MRP-Modul ist ein Teil der \textit{G3 HIS} Software, bei der bereits ältere Versionen im Einsatz sind. Dies macht die Software für Krankenhäuser interessant, welche ältere Versionen dieser Software verwenden oder ein anderes KIS besitzen.