\chapter{Exposee}

\section{Zielsetzung}
Das Modul MRP errechnet ideale Termine für Operationen in einem Krankenhaus unter der Berücksichtigung aller beteiligter Ressourcen ( Personen,Geräte , Material) und aller vor und nachgelagerten Untersuchungen sowie Bett und Zimmer. Ziel dieses Projektes ist es für alle Ressourcen bezogenen Daten (z.B. Dienste von Ärzten , OP Termin mit spezifischen Ärzten) eine Möglichkeit zu schaffen diese Termine in den persönlichen Kalendern der Personen (Teams) anzuzeigen damit die Anwender in ihrer gewohnten Umgebung die Ergebnisse des komplexen Planungsprozesses angezeigt bekommen. Darüber hinaus wäre eine Darstellung auf mobilen Geräten möglich. Programmiert wird in Java EE und AngularJs.
 

\section{Motivation}
Grundsätzlich wird davon ausgegangen, dass Termine im Outlook eingetragen werden. Dies sollte automatisch nach der Generierung von vorgegeben Terminvorschlägen geschehen. Diese Terminvorschläge werden als Teil des MRP-Systems generiert. Meine Aufgabe ist es mit Hilfe der Java EWS API die Daten an das Outlook zu senden bzw. an den Server übermitteln. Die EWS API erleichtert das Arbeiten mit Outlook enorm bei relativ flexibler API welche sowohl mit Java als auch mit C++ und XML verwendet werden kann.  \\

Diese Termine stellen dabei eine visuelle Darstellung dar. Änderungen von Terminen haben keine Auswirkung auf das MRP System  \\ 

Weiters können sogenannte "`Appointments"' mit Hilfe von vordefinierten Filtern gesucht und exportiert werden. Dies ermöglicht eine einfachere Handhabung von Anfragen. Diese Appointments werden dann in Outlook-Termine umgewandelt und können so exportiert werden. 


\section{Java EWS API}
Die \textit{ EWS Managed API} ist eine empfohlene Schnittstelle für die Entwicklung von Client - Anwendungen welche mit Exchange Services kommunizieren. Die API besteht aus dem Exchange Service, welcher als zentrale Kernkomponente angesehen wird.\\

Im wesentlichen besteht die API aus dem Anlegen und Verwalten von zeitbehafteten Terminen. Emails sowie Termine können in jedes Postfach eingetragen sowie abgerufen werden. Die EWS API unterstützt zeitgesteuerte Abfragen und Verfügbarkeiten, welche bereits Verfügbarkeitsvorschläge enthalten. Diese Vorschläge beziehen sich auf Personen- und Raumverfügbarkeiten 